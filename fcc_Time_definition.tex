\section{Definitions used in the studies}
\label{sec:Def_timing}
\subsection{Timing}
At the first place, the definition of the timing should be obtained as our benchmark to be the variable of distinguishing different number of subjet(s) in a large raduis jet. The standard formula of the timing is as follows for each particle:
\begin{equation}
 Timing = \frac{L(\eta)}{V(V_{x},V_{y},V_{z})}  
\end{equation}
Basically, it is the normal formula of the time of flight(TOF), depending on the different \(\eta\), leading to the different distances\(\ L\) between collision point and HCAL barrel, along with the three-dimensional velocities\(\ V \) of the particles.\\

But, since the effect of the magnetic field is taken into account, the timing used in this paper is the modified one with only considering the Z direction of the distance and velocity, and then we can get more precise value of the timing by the truth of the Z-direction trace of the particle isn't changed by the magnetic field. The formula used in this paper is as follows: 
\begin{equation}
 Timing = \frac{L_{z}}{V(V_{z})} 
\end{equation}

After applying this formula on all particles, and the timing can be used as a great variable for distinguishing many things in the detector. 

\subsection{Trailing particle - PT and T}
The definition of the trailing particles defined by the Timing(T) and transverse momentum(PT) should be clarified as well. We have two categories of the trailing particles
\begin{enumerate}
\item Defined by the timing, we called it "trailing-T", meaning that the particle has the longest traveling timing in the jet. 
\item Defined by the PT, we called it "trailing-PT", meaning that the lowest-PT particle in the jet. 
\end{enumerate}
Consequently, we can define next-to trailing particle as well, such as the second long traveling timing-"next-to-trailing-T" or the second low PT-"next-to-trailing-PT", and so on.
%%%%%%%%%%%%%%% commented out 
\end{comment}
