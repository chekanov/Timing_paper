\section{Studies of the components of the jet ( Include pre-selection )}
\label{sec:Def_timing}

Since the events of the HCAl barrel would be used as our candidates in order to avoid mixing up with the events flying to the endcap, and also the consistent of calculating the timing as well,  \(\eta <1\) cut is used to be our first pre-selection.\\ 

Considering the effect of the magnetic field is another issue that we should recognize. Since 5T magnetic field is applied, we can figure out the minimum PT that particle should have to arrive the HCAL barrel. In the end, 1.5GeV cut is applied as our second pre-selection and you can see there are a little bit pions cut off because of their low PT.\\

Theoretically, in our cases of WW and qq fragmentation, they should have the same fraction of the different kinds of particles in their large radius jet since both of them have the quark(s) be their final state. We've done checking the fractions of the different particles in these two processes. Please look at the picture~\ref{Jet_com}.\\

According to the plots, before and after the \(\eta\)\ cut, the fractions of the different particles are the same, and after cutting off the low-PT particles, the fractions don't change much but lose few pions (Relatively, the photon goes up). It is the expected circumstance and tells us that the jets could be used in our studies.

\begin{figure}
\begin{center}
   \subfigure[5TeV] {
   \includegraphics[width=0.45\textwidth]{/Users/ms08962476/singularity/TIming_Studies/Codes/5TeV/Try_Trailing_ID_Eta_cut_no_with_withPT_5TeV.pdf}
   }
   \subfigure[10TeV] {
   \includegraphics[width=0.45\textwidth]{/Users/ms08962476/singularity/TIming_Studies/Codes/10TeV/Try_Trailing_ID_Eta_cut_no_with_withPT_10TeV.pdf}
   }
   \subfigure[20TeV] {
   \includegraphics[width=0.45\textwidth]{/Users/ms08962476/singularity/TIming_Studies/Codes/20TeV/Try_Trailing_ID_Eta_cut_no_with_withPT_20TeV.pdf}
   }
    \subfigure[40TeV] {
   \includegraphics[width=0.45\textwidth]{/Users/ms08962476/singularity/TIming_Studies/Codes/40TeV/Try_Trailing_ID_Eta_cut_no_with_withPT_40TeV.pdf}
   }
\end{center}
\caption{The fractions of the jet components are shown to see the validation of the study.}
\label{Jet_com}
\end{figure}

%%%%%%%%%%%%%%% commented out 
\end{comment}
