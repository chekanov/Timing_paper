\section{Timing layers for FCC and jets}
In the previous sections, the capability of the timing among the various resolutions of the detector used on distinguishing the different kinds of single particle that has the different mass, momentum as well as the length of going through the detector is well-investigated as the excellent variable within the three sigma hypothesis. The next necessary step is to employ the timing, which is the potential method of discriminating the different number of the subjets in a large radius jet as well, into tackling the dilemma that has been being concerned about is the highly-boosted circumstances will lead the particles of the jet to too close to each other, and the truth number of the subjet could be misestimated, along with impacting on the analysis in the future.\\ 

The simplified-cases simulating the pileup-free conditions with the already-known processes are involved in these studies under the environment of a very-high-energy collider will be explored as the benchmark of the timing applied in the future. The same processes of Z'$\rightarrow$qq(Background), Z'$\rightarrow$WW(Signal) with 5, 10, 20 and 40TeV center-of-mass energy(C.M.) as the ones exploited in our second paper, are taken into account as the targets of doing the researches on the same matter with the timing implemented.\\ 
%adopting the jet-substructure variables to deal with estimating the internal structure of the boosted jets for those processes

In terms of digging out the potential of the timing, several studies have been done trying on a couple of variables to see whether the timing can give us another degree of information helping on having the improvement on the issue in addition to the $P_{T}$ that we could obtain directly from the detector . The $\Delta R$, which is the common variable that have been being used in the collider as seeing the angle between two particles/jets, is found to be the possible one that can take advantage of using the timing to analyze the structure of the jets more evidently. Also, the ideal cases of separating the different particles with the help of the timing are also premeditated.\\

The two sets of the collections of the data including both of the generator level(truth-level) and the reconstruction level(reco-level) are utilized in the studies for the purpose of corresponding between each other as the benchmark of the parameters that are found beneficial. At the first place, the definition of many terms applied in the studies will be well-defined as a favor of proceeding smoothly. The second one is to make use of the truth-level information to figure out the limitation of the timing being capable of helping on the issue at best ideally. Last but not least, the well-established cases done by using the reco-level information will be given as the true cases we could expect in our life when the time of the timing-capable detector installed in the collider with the very high C.M. energy comes.\\

\subsection{The definitions of the terms introduced in the following studies}
\label{sec:Def_timing}
In case for avoiding being confused by the terms applied in these studies, pre-defined on those terms are essential.\\
\subsubsection{The timing used in the truth-level cases}
At the first place, the definition of the timing should be obtained as our benchmark to do these studies with the truth-level information. The standard formula of the timing is as follows for each particle:
\begin{equation}
 Timing = \frac{L(\eta)}{V(V_{x},V_{y},V_{z})}  
\end{equation}

Basically, it is the normal formula of the time of flight(TOF), depending on the different \(\eta\), leading to the different distances\(\ L\) between collision point and HCAL barrel, along with the three-dimensional velocities\(\ V \) of the particles.\\

But, since the effect of the magnetic field is taken into account, the timing applied in this paper is the modified one with only considering the Z direction of the distance and velocity, and then we can get more precise value of the timing by the truth of the Z-direction trace of the particle isn't changed by the magnetic field. The formula used in this paper is as follows: 

\begin{equation}
 Timing = \frac{L_{z}}{V(V_{z})} 
\end{equation}

After applying this formula on all particles, and the timing can be obtained without the bending effect coming from the magnetic field.\\ 
\subsubsection{The trailing particles applied in all cases}
The another terms should be noticed as the important ones are two sorts of the trailing particles applied in all cases. The definition of the trailing particles defined by the Timing(T) and transverse momentum(PT) should be clarified. We have two categories of the trailing particles
\begin{enumerate}
\item Defined by the T, so-called "trailing-T", meaning that the particle has the longest traveling timing in the jet. 
\item Defined by the PT, so-called "trailing-PT", meaning that the lowest-PT particle in the jet. 
\end{enumerate}

Logically, we can define next-to trailing particle as well, such as the second long traveling timing-"next-to-trailing-T" or the second low PT-"next-to-trailing-PT", and so on.

%%%%%%%%%%%%%%% commented out 
%\end{comment}

