\section{Benefit of the dR with the timing}
\label{sec:Def_timing}
In this section, the goal is to explore the capability of the timing that can help to distinguish the different number of subjet(s) in the large radius jet by using the PT-trailing particles series alone first, and then adding the "timing information"-T-trailing particles series subsequently, to see whether the timing could be taken as our advantage. The dR between the highest PT particle and the other trailing particles in the big radius jet at 5TeV are shown as an example in the figure~\ref{dR_5TeV}.\\

The trailing particles to the next-to-next-to-next-to-next-to trailing particles are the scenarios for us to find out the benefit of the timing at the different CM energies. The figure~\ref{BDT} show that such low CM energy as 5TeV can be found that the timing could help us have a little improvement on the background rejection. But when going into the higher and higher CM energy as 40TeV, the PT-trailing of dR is enough for us to distinguish different number of subjets in this case. 

\begin{figure}
\begin{center}
   \subfigure[Trailing-PT] {
   \includegraphics[width=0.45\textwidth]{/Users/ms08962476/singularity/TIming_Studies/Codes/5TeV/Try_dR_PT_0_5TeV.pdf}
   }
   \subfigure[Trailing-T] {
   \includegraphics[width=0.45\textwidth]{/Users/ms08962476/singularity/TIming_Studies/Codes/5TeV/Try_dR_T_0_5TeV.pdf}
   }
   \subfigure[Next-to-trailing-PT] {
   \includegraphics[width=0.45\textwidth]{/Users/ms08962476/singularity/TIming_Studies/Codes/5TeV/Try_dR_PT_1_5TeV.pdf}
   }
   \subfigure[Next-to-trailing-T] {
   \includegraphics[width=0.45\textwidth]{/Users/ms08962476/singularity/TIming_Studies/Codes/5TeV/Try_dR_T_1_5TeV.pdf}
   }
\end{center}
\caption{The 5TeV plots are chosen as the example of the dR plots. }
\label{dR_5TeV}
\end{figure}

\begin{figure}
\begin{center}
   \subfigure[5TeV] {
   \includegraphics[width=0.45\textwidth]{/Users/ms08962476/singularity/TIming_Studies/Codes/5TeV/BDT_plot_dR_dRplusID_5TeV.pdf}
   }
   \subfigure[10TeV] {
   \includegraphics[width=0.45\textwidth]{/Users/ms08962476/singularity/TIming_Studies/Codes/10TeV/BDT_plot_dR_dRplusID_10TeV.pdf}
   }
   \subfigure[20TeV] {
   \includegraphics[width=0.45\textwidth]{/Users/ms08962476/singularity/TIming_Studies/Codes/20TeV/BDT_plot_dR_dRplusID_20TeV.pdf}
   }
   \subfigure[40TeV] {
   \includegraphics[width=0.45\textwidth]{/Users/ms08962476/singularity/TIming_Studies/Codes/40TeV/BDT_plot_dR_dRplusID_40TeV.pdf}
   }
\end{center}
\caption{The BDT plots of the dR histograms from trailing to the next-to-next-to-next-to-next-to trailing particles are shown to see the benefit of the timing. The blue line refers to the PT-trailing series only case, and the black line shows the condition as combining the PT and T-trailing series together.}
\label{BDT}
\end{figure}

%%%%%%%%%%%%%%% commented out 
\end{comment}
